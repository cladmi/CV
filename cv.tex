% CV fictif destine à servir de modèle à la classe de curriculum vitae cv.cls
% Le fichier cv.cls doit se trouver dans le même repertoire ou dans un repertoire
% accessible par LaTeX (voir l'utilisation de la variable TEXINPUTS).
% 5 Fevrier 2003 -- Frederic Meynadier (Frederic.Meynadier@obspm.fr)
%
%
\documentclass{cv}

\usepackage[francais]{babel}
\usepackage[utf8]{inputenc}
\usepackage[T1]{fontenc}       %% Pour la cesure des mots accentues
\usepackage[paper=a4paper,textwidth=160mm]{geometry}
\usepackage{eurosym}

\textheight=28cm
\textwidth=18.5cm
\hoffset = -1.25cm
\voffset = -1.5cm
%\headsep = 1cm
%\headheight = 0cm
\topmargin = -1cm

\newcommand{\lieu}[1]{\textsl{#1}\ }
%\newcommand{\lieu}[1]{{#1}\ }
\newcommand{\activite}[1]{\textbf{#1}\ }
\newcommand{\projet}[2]{{#1} \textbf{#2}\ }
%\newcommand{\comment}[1]{\textsl{#1}\ }
\newcommand{\comment}[1]{{#1}\ }
\newcommand{\group}[1]{\hspace{1em}\textsl{#1}\ }
\newcommand{\hs}{\hspace{1.6em}}


\begin{document}
\begin{chapeau}
\hspace{1cm}
\begin{adresse}
        Gaëtan HARTER\\
        38, Avenue Maréchal Randon\\
        38\,000 Grenoble\\
        \vspace{3pt}
        Tel.~:\,06~85~33~33~22\\
        E-mail~:\,\texttt{hartergaetan@gmail.com}
        Github~:\,\texttt{https://github.com/cladmi}
\end{adresse}
\begin{etatcivil}
        Né le 13 Janvier 1989\\
        Nationalité Française\\
        Mobilité: Grenoble\\
        Permis B
\end{etatcivil}
\hspace{0.5cm}
\end{chapeau}
\vspace{0.5cm}
\begin{center}
        \textbf{\Large Ingénieur Systèmes et logiciels embarqués\\}
        \textbf{\large Grenoble INP - Ensimag\\}
        \vspace{0.5em}
        \textbf{\large Recherche d'un emploi en Systèmes embarqués\\} %\\Intérêt pour les méthodes agiles\\}
        {À partir de septembre 2017}
\end{center}



        %%%%%%%%%%%%%%%%%%
        % Bloc rubriques %
        %%%%%%%%%%%%%%%%%%

%
%
%        Liste des Emplois/réalisés
%
%
\begin{rubriquetableau}[2.5cm]{Activités professionnelles}

2011~--~2017            & \activite{Architecture et développement plateforme de réseau de capteurs FIT-IoT-LAB}\\
\group{1$^{er}$ emploi}
                        & \hs Conception et développement de l'architecture logicielle, système et réseau de la plateforme.\\
~                       & \hs Génération de distribution embarquée Linux Yocto pour un déploiement large échelle NFS.\\
~                       & \hs Développement embarqué d'applications d'administration, librairies et exemples utilisateur.\\
~                       & \hs Conception et développement d'outils utilisateurs d'accès et d'automatisation.\\
~                       & \hs Documentation en anglais et support utilisateur au développement embarqué et outils.\\
~                       & \hs Implémentation d'un systême d'intégration continue complet, de la génération de la distribution, son déploiement et des tests sytèmes sur cible.\\
~                       & \hs \texttt{C, Python, ARM-CortexM3, RTOS, Linux Embarqué - Yocto, IPv6, 6LowPAN, 802.15.4, intégration continue}.\\
~                       & \lieu{\hs Équipe FIT-IoT-LAB - Inria Grenoble}\\

2015, 2016              & \activite{Ensimag - Encadrement projet de développement d'un système d'exploitation x86}\\
% 2013, 2014              & \activite{Ensimag - Encadrement projet de développement en C}\\
    \group{Enseignement} &

~ \\
2011                    & \activite{Stage Fin d'étude - Sauvegarde externe de contexte pour récepteur satellite}\\
                        & \comment{
                        \hs Génération d'un core dump processeur et mémoire rechargeable en débuggeur.
                        \texttt{Architecture CPU et mémoire, Assembleur, C++, gdb, ldscript}.
                        % Travail avec des méthodes agiles. Équipement certifié DO-178B.
                        }\\
~                       & \lieu{\hs Équipe de développement logiciel Galileo - THALES Avionics Valence}\\

%2010\textsl{ - 3 mois}  & \activite{Communications P2P multi-sauts pour iPhone OS}\\
%\group{Stage}           & \comment{
%                        \hs Réalisation d'un framework de communication autoconfiguré via géo-routage pair à pair.
%
%                        \hs Réseau~Ad-Hoc, Objective-C, Bonjour, AsyncSocket} \\
%~                       & \lieu{\hs Équipe Drakkar - Laboratoire Informatique de Grenoble}\\

%2010\textsl{ - 2 mois}  & \activite{Travail~:\,Application Twitter-like pour iPhone}
%                        \comment{\newline Travail seul. Conception et réalisation du client sur iPhone et du serveur dédié en java.}\\

% 2010\textsl{ - 1 mois} & \activite{Compilateur en ada d'un sous langage de Java}\\
% \group{Quadrinôme}     & \comment{\hs Analyse lexicale, syntaxique et génération de code}\\
%

\end{rubriquetableau}


\begin{rubriquetableau}[2.5cm]{Formation}
2008~--~2011
                        & \activite{Grenoble INP - Ensimag - Ingénieur systèmes et logiciels embarqués
                        \newlineÉcole nationale supérieure d'informatique et de mathématiques appliquées}
                        \comment{\newline Système d'exploitation, temps réel, programmation concurrente et parallèle, systèmes sur puce, systèmes distribués, sécurité matérielle, tolérance aux fautes} \\

2006~--~2008            & \activite{Classe préparatoire aux Grandes Écoles - MP* Informatique} \lieu{ Lycée Kleber, Strasbourg}
%2006~--~2008
%                        & \activite{Classe préparatoire aux Grandes Écoles} \lieu{ Lycée Kleber, Strasbourg (67)}
%                        \comment{\newline Mathématiques, physique, sciences de l'ingénieur et informatique} \\
%2006
%                        & \activite{Bac Scientifique -- Mathématiques et sciences de l'ingénieur -- Mention Bien}
\end{rubriquetableau}



%
%        Projets représentatifs de mes attentes
%
\begin{rubriquetableau}[2.5cm]{Projets scolaires représentatifs}
\activite{OS embarqué}  & Système d'exploitation multitâche sur CPU MIPS - \textsl{C, assembleur} \\
\activite{Driver Linux} & Pilote de carte réseau FPGA pour Linux embarqué - \textsl{C, Linux embarqué} \\
\activite{Distribué}    & Table de hachage distribuée - \textsl{C, socket.h, concurrence, distribué} \\

%2010\textsl{ - 1 mois}  & \activite{Table de hachage distribuée} \\
%\group{Trinôme}         & \comment{\hs Conception et développement en C d'une DHT avec répartition des serveurs en anneau.}\\
%
%2010\textsl{ - 4 mois}  & \activite{Pilote de carte réseau pour un Linux embarqué sur carte programmable FPGA} \\
%\group{Binôme}          & \comment{\hs Développement d'un pilote linux pour un contrôleur réseau sans documentation du matériel.}\\
%
%
%
%2010\textsl{ - 1 mois}  & \activite{Système d'exploitation à temps partagé pour une carte programmable FPGA}\\
%\group{Binôme}          & \comment{\hs Réalisation en C d'un OS embarqué gérant la concurrence et séparation noyau/utilisateur.}\\
\end{rubriquetableau}


\begin{rubriquetableau}[2.5cm]{Compétences en informatique}
\activite{Langages~:}   & \comment{C, Python, Assembleur, Bash}\\
                        %& \comment{Notions~:\,C++, Objective-C~-~iOS, Java, \LaTeX, {SystemC~-~TLM}, VHDL}\\
\activite{Outils~:}     & \comment{GNU/Linux, Git avancé, Yocto, Jenkins, Terminal/vim, Makefile}\\
\activite{Tests~:}      & \comment{Tests unitaires et d'intégration, mise en place d'un système d'intégration continue}
\end{rubriquetableau}

\begin{rubriquetableau}[2.5cm]{Langues}
\activite{Anglais}      & \comment{Technique et courant à l'écrit et l'oral, TOEIC 835}\\
\activite{Allemand}     & \comment{Bonne compréhension orale. Niveau A2}
\end{rubriquetableau}

\begin{rubriquetableau}[2.5cm]{Loisirs}
                        & Musique métal, Jeux vidéos, Cuisine, Escalade\\
                        & \activite{Engagement associatif}\\
2015~--Actuel           & \comment{\hs Trésorier de JustBeer - Association Grenobloise de dégustation de bière}\\
2015~--Actuel           & \comment{\hs Trésorier de Mormegil - Association Grenobloise d'organisation de concert de type métal}\\
%2010~--~2011            & \comment{\hs\textbf{Responsable Achats} et qualité (CA 260k\euro) -- Bureau des Élèves de Grenoble~INP}\\
%2009~--~2010            & \comment{\hs\textbf{Président Cafétéria} Préparation quotidienne de sandwichs (40-70 personnes)} -- Ensimag\\
%                        & \comment{\hs\textbf{Organisateur} de repas mensuels (100 à 200 personnes) -- Bureau des Élèves de l'Ensimag}\\
%                        & \comment{\hs\textbf{Vice-Président} -- Club de dégustation de bière de Grenoble~INP}\\
%                        & \activite{Escalade}\\
                        %& \activite{Baby-foot}\\
                        %& \activite{Jeu de rôle}\\
                        %& \activite{Mangas}\\
                        %& \activite{Penspinning}
\end{rubriquetableau}

\begin{center}
        \vspace{0.5em}
        \textbf{Références disponibles sur demande}
\end{center}

\end{document}
