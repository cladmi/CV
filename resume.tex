% CV fictif destine à servir de modèle à la classe de curriculum vitae cv.cls
% Le fichier cv.cls doit se trouver dans le même repertoire ou dans un repertoire
% accessible par LaTeX (voir l'utilisation de la variable TEXINPUTS).
% 5 Fevrier 2003 -- Frederic Meynadier (Frederic.Meynadier@obspm.fr)
%
%
\documentclass{cv}

\usepackage[english]{babel}
\usepackage[utf8]{inputenc}
\usepackage[T1]{fontenc}       %% Pour la cesure des mots accentues
\usepackage[paper=a4paper,textwidth=160mm]{geometry}
\usepackage{eurosym}

\textheight=28cm
\textwidth=18.5cm
\hoffset = -1.25cm
\voffset = -1.5cm
%\headsep = 1cm
%\headheight = 0cm
\topmargin = -1cm

\newcommand{\lieu}[1]{\textsl{#1}\ }
%\newcommand{\lieu}[1]{{#1}\ }
\newcommand{\activite}[1]{\textbf{#1}\ }
\newcommand{\projet}[2]{{#1} \textbf{#2}\ }
%\newcommand{\comment}[1]{\textsl{#1}\ }
\newcommand{\comment}[1]{{#1}\ }
\newcommand{\group}[1]{\hspace{1em}\textsl{#1}\ }
\newcommand{\hs}{\hspace{1.6em}}


\begin{document}

% Header
\begin{chapeau}
\hspace{1cm}
\begin{adresse}
        Gaëtan HARTER\\
        French \\
        Living in Berlin
\end{adresse}
\begin{etatcivil}
        Phone~:\,+33~6~85~33~33~22
        Mail~:\,\texttt{hartergaetan@gmail.com}
        Github~:\,\texttt{https://github.com/cladmi}
\end{etatcivil}
\hspace{2cm}
\end{chapeau}

% Main description
%\vspace{0.5cm}
\begin{center}
        \textbf{\Large Embedded system and software engineer\\}
        \large 9 years work experience\\

        \vspace{0.5em}
        \begin{minipage}{15cm}
          \begin{center}
            \large I enjoy low level concurrent code, write atomic commits, care about tooling and improve development with automated CI testing.
          \end{center}
        \end{minipage}
        %\\Intérêt pour les méthodes agiles\\}
        %\vspace{0.5em}{From September 2017}
\end{center}

%%%%%%%%%%%%%%%%%%
% Bloc rubriques %
%%%%%%%%%%%%%%%%%%
%
%
%        Liste des Emplois/réalisés
%
%
\begin{rubriquetableau}[2.5cm]{Work experience}
2020
                        & \activite{Software quality engineer for safety critical embedded system}\lieu{MSA Safety Berlin}\\
~                       & \hs Review code and tests on SIL3 certified software. Emphasize role of code structure and readability.
                              Develop tools and test automation. Mentoring on software design, architecture.\\
~                       & \hs \texttt{C, Python, git, Bamboo, SIL 3, VectorCAST}\\
~\\

2017~--~2019
                        & \activite{Rapstore - App store for the open source embedded OS RIOT}\lieu{Freie Universität Berlin}\\
~github:\texttt{RIOT-OS}
                        & \hs Implement support for external software and boards development, simplify hardware testing.
                              Support firmware update. Build system maintainer. Open source project maintainer.\\
~                       & \hs \texttt{C, Python, bash, Real Time OS, git, make, pytest, Jenkins, github}\\
~\\

2011~--~2017
                        & \activite{Embedded development for large scale sensor testbed FIT-IoT-LAB}\lieu{Inria Grenoble}\\
~github:\texttt{iot-lab}
                        & \hs Design and develop the embedded software and IPv6 network architecture.
                              Experiment control in Python, realtime monitoring on FreeRTOS/stm32f3,
                              Yocto distributions, large-scale deployment over NFS.
                              Implement on target unit and integration testing with 100\% branch coverage. \\
~                       & \hs Users support and development of users experimentation tools and firmwares.\\
~                       & \hs \texttt{C, Python, ARM-CortexM3, RTOS, Yocto, IPv6, Jenkins, continuous integration}\\
~\\

\textsl{2~years} & \activite{Teaching/Project: Develop an OS with virtual memory on x86}\lieu{Grenoble INP - Ensimag}\\
%~\textsl{2015, 2016}
%~                       & \lieu{\hs Grenoble INP - Ensimag}\\
% 2013, 2014              & \activite{Development project of a MIPS assembler}\\
% \group{Teaching}
% ~                       & \lieu{\hs Grenoble INP - Ensimag}\\

Internship               & \activite{Post assertion core dump solution for a embedded system DO-178B}\lieu{THALES}\\
%~\textsl{2011-6~month}
%                        & \hs Generate a full post assertion processor and memory state core dump and debugger reloading\\
%~                       & \hs \texttt{DSP asm, gdb, linkerscript, COFF} \lieu{Galileo Positioning System ground equipments}
%~                       & \lieu{\hs Galileo Positioning System ground equipments software team - THALES Avionics Valence}\\
%
%2010\textsl{ - 3 mois}  & \activite{Communications P2P multi-sauts pour iPhone OS}\\
%\group{Stage}           & \comment{
%                        \hs Réalisation d'un framework de communication autoconfiguré via géo-routage pair à pair.
%                        \hs Réseau~Ad-Hoc, Objective-C, Bonjour, AsyncSocket} \\
%~                       & \lieu{\hs Équipe Drakkar - Laboratoire Informatique de Grenoble}\\
\end{rubriquetableau}


\begin{rubriquetableau}[2.5cm]{Education}
%2008~--~2011            & \activite{Master degree in embedded systems and softwares engineering} \lieu{Grenoble INP - Ensimag}\\
2008~--~2011            & \activite{Master in embedded systems and softwares engineering} \lieu{Grenoble INP - Ensimag} \\
%~                       & \hs \comment{ASM/C/Object oriented, concurrent systems, operating system, Linux device driver}\\
2006~--~2008            & \activite{Highly selective classes preparing for competitive exams} \lieu{Prépa Kleber}
\end{rubriquetableau}

% \begin{rubriquetableau}[2.5cm]{Representative school projects}
% \activite{Embedded OS}  & Embedded MIPS multitask operating system with kernel/user space - \textsl{C, assembly, syscalls, irq} \\
% \activite{Linux Driver} & FPGA network card driver on an embedded Linux - \textsl{C, Embedded Linux} \\
% \activite{Distributed}  & Distributed hash table design and development - \textsl{C, unix sockets, concurrency, distributed} \\

%2010\textsl{ - 1 mois}  & \activite{Système d'exploitation à temps partagé pour une carte programmable FPGA}\\
%\group{Binôme}          & \comment{\hs Réalisation en C d'un OS embarqué gérant la concurrence et séparation noyau/utilisateur.}\\
%\end{rubriquetableau}


\begin{rubriquetableau}[2.5cm]{Technical skills}
\activite{Languages}    & \comment{Low level concurrent C, Python, Bash, Assembly}\\
                       %& \comment{Notions~:\,C++, Objective-C~-~iOS, Java, \LaTeX, {SystemC~-~TLM}, VHDL}\\
\activite{Tools}        & \comment{GNU/Linux, Advanced git, Yocto, Jenkins, GNUMake, Terminal fluent, vim}\\
\activite{Testing}      & \comment{Unit and integration tests, setup of an on target continuous integration system}
\end{rubriquetableau}

\begin{rubriquetableau}[2.5cm]{Languages}
\activite{French}       & \comment{Mother tongue}\\
\activite{English}      & \comment{Fluent. Working in English since 2011}\\
\activite{German}       & \comment{Conversational, used regularly, Level B2 in 2019}
\end{rubriquetableau}

\begin{rubriquetableau}[2.5cm]{Activities}
\activite{Free Time}    & Music, cooking, video games, climbing, beer tasting\\

%\activite{Volunteering} & \comment{Treasurer: Beer tasting and concert production non profit organizations}\\
%~                       & \comment{University: In charge of the Cafeteria, Food events organization and purchasing expert}
% 2015~--Now              & \comment{\hs Treasurer of JustBeer - Beer tasting organization}\\
% 2015~--Now              & \comment{\hs Treasurer of Mormegil - Metal concert production organization}
%2010~--~2011            & \comment{\hs\textbf{Responsable Achats} et qualité (CA 260k\euro) -- Bureau des Élèves de Grenoble~INP}\\
%2009~--~2010            & \comment{\hs\textbf{Président Cafétéria} Préparation quotidienne de sandwichs (40-70 personnes)} -- Ensimag\\
%                        & \comment{\hs\textbf{Organisateur} de repas mensuels (100 à 200 personnes) -- Bureau des Élèves de l'Ensimag}\\
%                        & \comment{\hs\textbf{Vice-Président} -- Club de dégustation de bière de Grenoble~INP}\\
\end{rubriquetableau}

\begin{center}
        \vspace{0.5em}
    \textbf{References available upon request.}
\end{center}

\end{document}
